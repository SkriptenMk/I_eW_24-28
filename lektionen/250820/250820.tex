\documentclass[a4paper]{scrreprt}
\usepackage[german]{babel}
\usepackage[utf8]{inputenc}
\usepackage{array}
\usepackage{booktabs}
\usepackage{longtable}
\usepackage{ragged2e}
\usepackage{enumitem}
\setlist[itemize]{noitemsep}

\begin{document}
\section*{1eW, 20. August 2025, Introduction to Computer Networks}
\begin{longtable}{p{1.5cm}>{\RaggedRight}p{7.5cm}p{2.5cm}}
    \toprule
    \emph{Zeit}&\emph{Inhalt}&\emph{Methode}\\
    \midrule
    \endhead

    \midrule
    \multicolumn{3}{c}{\begin{tiny}\textit{to be continued}\end{tiny}}\\
    \midrule
    \endfoot

    \bottomrule
    \endlastfoot

    1420&Term Schedule&\\ [5pt]

    1425&What happens when calling an internetadress?&Lehrgespräch\\
        &Translation of the domain name to an IP-address&\\
        &nslookup www.nzz.ch (on the default DNS Server)&\\
        &nslookup www.nzz.ch 1.1.1.1&\\
        &nslookup www.nzz.ch 9.9.9.9&\\
        &Comparison of the results&\\ [5pt]

    1450&IPv4 in comparison to IPv6&Lehrgespräch\\
        &Display an number of daddresses&\\ [5pt]

    1500&local IP-address&Lehrgespräch\\
        &ipconfig&\\
        &Discussion of the result (including the subnetmask)&\\ [5pt]

    1515&External IP-address&Lehrgespräch\\
        &https://whatismyip.org&\\ [5pt]

    1520&NAT&Lehrgespräch\\ [5pt]
        &Direct call of Google through the IP-address with port 80&\\ [5pt]

    1530&OSI respectively TCP/IP Layer model&Lehrvortrag\\ [5pt]

    1540&ev. installation of Wireshark&Lehrvortrag\\




\end{longtable}
\end{document}
